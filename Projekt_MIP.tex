\documentclass[10pt,oneside,slovak,a4paper]{article}

\usepackage[slovak]{babel}
\usepackage[T1]{fontenc}
\usepackage[IL2]{fontenc} 
\usepackage[utf8]{inputenc}
\usepackage{graphicx}
\usepackage{url} 
\usepackage{hyperref} % odkazy v texte budú aktívne (pri niektorých triedach dokumentov spôsobuje posun textu)

\usepackage{cite}
\usepackage{times}

\pagestyle{headings}

\title{Agilné vývojové metódy vs tradičné vývojové metódy\thanks{Semestrálny projekt v predmete Metódy inžinierskej práce, ak. rok 2021/22, vedenie: Ing. Fedor Lehocki, PhD.}}

\author{Juraj Majerník\\[2pt]
	{\small Slovenská technická univerzita v Bratislave}\\
	{\small Fakulta informatiky a informačných technológií}\\
	{\small \texttt{xmajernik@stuba.sk}}
	}

\date{\small 28.10. 2021}

\begin{document}
\maketitle
\begin{abstract}
\centering
Konkurencia medzi IT spoločnosťami vyžaduje od nich dodanie softvérových riešení svojmu 
klientovi v čo najkratšiom čase, pričom požiadavky klienta sa môžu v procese vývoja zmeniť. 
Jednými z hlavných dôvodov týchto zmien sú, že zákazník na začiatku jednania nemusí 
presne vedieť čo chce od softvérového riešenia, alebo sa počas vývoja objavila lepšia 
možnosť, ako daný problém vyriešiť. V tomto článku si priblížime agilné a klasické metódy a 
porovnáme ich základné princípy, robustnosť, flexibilitu a náročnosť z časového 
a finančného hľadiska. Dozvieme sa, s akými komplikáciami sa musia inžinieri pri jednotlivých 
metódach vysporiadať a či by agilná metóda mohla nahradiť ostatné vývojové metódy.
\end{abstract}

\section{Úvod}
Témou práce sú agilné metódy vs tradičné vývojové metódy v softvérovom inžinierstve. V dnešnej dobe je táto problematika veľmi konzultovanou témou. 
V našej práci chceme zistiť dôvod, pre ktorý ešte stále nenahradili agilné metódy tradičné metódy, keďže sú efektívnejšie a flexibilnejšie. Flexibilita je zároveň v dnešnej dobe vyhľadávaná schopnosť, keďže nám dovoľuje sa prispôsobiť zmenám od klientov a tým nám pomáha šetriť čas a peniaze pri dodaní nášho riešenia. 
V prvej časti článku sa venujem agilnému\ref{Agilne} vývoju ako takému. Dozvieme sa tu, čo to agilný vývoj vlastne je, jeho výhody a nevýhody, prečo boli vyvinuté a načo sa používajú. 
V druhej časti článku sa venujem extrémnemu programovaniu\ref{XP}. Povieme si o jeho praktikách, výhodách a nevýhodách. 
V tretej časti článku si priblížime metódu vodopádového vývoja\ref{vodopad}. 
Vo štvrtej časti článku si povieme o špirálovom vývoji\ref{spiral}, o ich plusoch a mínusoch. 
V piatej časti článku sa dozvieme niečo o RAD vývoji\ref{RAD}. 
V poslednej – šiestej časti článku sa venujeme rozdielom medzi jednotlivými metódami\ref{porovnávanie} a porovnávame ich výhody a nevýhody. 
Cieľom práce je zistiť či sú agilné metódy vhodné na nahradenie všetkých zvyšných procesov. Predstavíme si tieto základné metodiky, popíšeme si ich základné informácie, vymenujeme výhody a nevýhody z pohľadu na flexibilitu, časovej náročnosti a robustnosti. V každej časti článku poukážeme na prípady softvéru, pre ktoré sú jednotlivé metódy vhodné a nevhodné a oddôvodnime si ich. V závere\ref{zaver} odpovieme na otázku či môžu nahradiť agilné metódy tradičné vývojové metódy. 



\section{Agilné vývojové metódy}\label{Agilne}
Agilné vývojové metódy, sú všetky metódy, ktoré používajú iteratívny a prírastkový vývoj a riadia sa podľa hodnôt a princípov agilného vývojového manifesta ktoré vzniklo v roku 2001. Autormi Agilného vývojového manifesta boli zástupcovia Extrémneho programovania, SCRUM, DSDM, Adaptive Software development, Crystal Feature driven development, Pragmatic Programming, ktorí chceli vytvoriť alternatívu k dovtedy používaním metódam\cite{Agile}, ktoré mali rigidný postup na základe dokumentácie, ako napríklad vodopádová metóda, pri ktorej ak sa vyskytla chyba v kóde pri testovaní, musel sa prerobiť celý kód. Agilné metódy sa snažili týmto nedostatkom predísť neustálou iteráciou kódu a, ako už sme si povedali agilné metódy sa riadia podľa agilného manifesta, ktoré je založené na 12-tích princípoch a 4 hodnotách, ktoré si teraz rozoberieme.
\\
\\
\textbf{Princípy Agilného manifesta}
\cite{Agile_principles}
\\
\\
\textbf{Hodnoty Agilného manifesta}
\begin{enumerate}
\item Ľudia a komunikácia sú viac než procesy a nástroje.
\item Funkčný softvér je viac než vyčerpávajúca dokumentácia.
\item Spolupráca s klientom je viac než dojednanie zmluvy.
\item Radšej reagovať na zmenu než sa držať plánu.\cite{Agile_Values}
\end{enumerate}

Podľa týchto hodnôt vieme povedať, že agilné metódy oproti tradičným metódam, ako je napríklad vodopádová metóda sa namiesto dokumentácie zameriava hlavne na priamu komunikáciu medzi všetkými zainteresovanými osobami v projekte a, že preferuje fungujúci softvér namiesto presnej dokumentácii, čo ma samo o sebe pár výhod, ako je napríklad to, že klient má prístup k rôznym verziám softvéru, ktoré môže on sám testovať a na základe týchto verzií nám počas vývoja môže povedať, že chce niečo pozmeniť alebo prišiel na nejakú novú funkciu, ktorú by sme mohli pridať a nám nerobí problém sa prispôsobiť jeho novým požiadavkám a pridať ich do ďalšej iterácie, čo nám šetrí čas a peniaze oproti tomu, ako by sme s klientom nekomunikovali počas celého vývojového procesu a na koniec by sme mu doručili softvér, ktorý nespĺňa úplne jeho požiadavky a museli ho prerobiť od základov, alebo v najhoršiom prípade by sme prišli o celú zákazku. Ďalej je tu priama komunikácia medzi vývojovými tímami, ktorej hlavná výhoda je, že si vedia pri priamej komunikácii efektívnejšie predať informácie medzi sebou a tým pádom zefektívnia svoju prácu. Bohužiaľ žiadna metóda nie je perfektná a agilné metódy majú aj svoje nevýhody. To, že preferujeme funkčný softvér namiesto dokumentácie je samo o sebe fajn, ale bez podrobnej dokumentácie nepoznáme konečný výsledok našej práce od 1. dňa a nevieme presne vyhradiť potrebné financie a koľko bude daný projekt trvať. Ďalej sa môžeme stratiť v neustálom pridávaní nových funkcionalít do programu, čím neustále zväčšujeme množstvo času a financií potrebných na vydanie finálnej verzie.


\subsection{Extrémne programovanie}\label{XP}
Extrémne programovanie alebo XP je ľahká agilná metóda, navrhnutá pre malé až stredne veľké projekty, pri ktorých sa očakáva, že sa bude kód často meniť. Zástancovia XP tvrdia, že zlepšuje kvalitu softvéru, zvyšuje produktivitu, znižuje náklady, je efektívny a nízko rizikový. Zameriava sa na rýchle cykly vydávania softvéru a nepretržitú komunikáciu s klientom, je založené na 12 praktikách, ktoré sa musia dodržiavať \cite{1008537}.

\textbf{Praktiky Extremného programovania pozostávajú z:}
Plánovacia hra,  Metafora XP,  Malé vydania, Jednoduchý dizajn, Testy jednotiek,  Refaktoring,  Párové programovanie, Kolektívne vlastníctvo,  Neustála integrácia, 40 hodinový týždeň, Zákazík priamo na mieste vývoja a Rovnaký kódovací štandard.\cite{1008537}\\


\textit{„XP stavia na osvedčených postupoch, ako je testovanie jednotiek, párové programovanie a refaktoring. V XP sú tieto praktiky kombinované tak, že sa navzájom dopĺňajú a často kontrolujú\cite{10.5555/1076267}.“}\\

Tento výrok nám poukazuje na jednu z najväčších nevýhod u XP. Každý programátor v tíme musí tieto praktiky nasledovať, pretože každá praktika má sama o sebe svoje nedostatky a tieto nedostatky sú vykryte inou praktikou. Ak by sme zobrali jednu z praktik preč začnú sa objavovať problémy. Zoberme si napríklad praktiku refaktoringu je to proces prepisu kódu s cieľom zjednodušiť  ho, aby sa dal jednoduchšie upravovať počas vývoja. Ak túto pratiku nebudeme vykonávať, tak kód, ktorý neustále upravujeme bude čím ďalej nečitateľnejší a nikto sa v ňom nebude vedieť orientovať a je jedno koľko ľudí na rozlúštenie kódu bude pracovať, pretože dokumentácia nie napísaná rozsiahla dokumentácia a nikto nebude vedieť, čo je, čo už len tým, že sme e nepoužili jednu praktiku sme prišli o peniaze, čas a pravdepodobne aj zákazníka, pretože sme mu nedodali softvérové riešenie včas. Ďalšou nevýhodou môže byť párové programovanie už len z toho hľadiska, že nie všetci programátori ho radi praktizujú a najradšej by podľa mňa boli sami pri jednej obrazovke a však má svoju výhodu čím viacej ľudí tvorí jeden kód tým menej chýb v kóde, plus komunikácia medzi ľuďmi z mojej vlastnej  skúsenosti napomáha k rozmýšľaniu. Ďalšiou výhodou je, že programátori musia nasledovať praktiku rovnakého kódovacieho štandardu, čo zapríčiňuje, že kód vyzerá ako, keby bol  napísaný jedným človekom a umožňuje každému z tímu pracovať na akejkoľvek časti kódu bez toho, aby musel strácať čas nad lúštením kódu, napísaného niekým iným. Praktika 40 hodinových týždňov zabezpečuje včasné dodanie sofvtvérového riešenia klientovi no treba dávať pozor, aby sa nestali normou, pretože unavený programátor robí častejšie chyby, ktoré sa budú musieť opravovať. 


\section{Vodopádový model}\label{vodopad}
Vodopádový vývojový model je SDLC, ktorý bol po prvýkrát definovaný Winstnom W. Roycom okolo roku 1970\cite{Sherrell2013}. Je to lineárny sekvenčný model, definovaný svojim lineárnym, štruktúrovaným prístupom k riadeniu projektov. Skladá sa zo série krokov, ktoré sú dokončené v sekvenčnom poradí.

Fázy vodopádového vývoja
\begin{enumerate}
\item Požiadavky - Urobí sa zápis  klientových požiadaviek.
\item Špecifikácie - Vytvorí sa oficiálny dokument, v ktorom sú spísané dané požiadavky.
\item Návrh -  Skladá sa z logického návrhu a fyzického návrhu.
\item Implementácia - V tomto kroku sa začne aj ukončí tvorba kódu.
\item Jednotkové Testovanie - Testovanie jednotlivých modulov.
\item zjednotené testovanie - Spojenie modulov a ich následné testovanie.
\item Údržba - Vykonávanie údržby kódu počas používania klientom.
\item Vyradenie - Systém je vyradený zo služby ak už nemôže byť udržiavaný alebo je zastaralý.\cite{Sherrell2013}
\end{enumerate}

Vodopádový model je jedným z najednoduchšich vývojových modelov, vďaka svojej štruktúre. Zároveň je to prvý Kladie sa v ňom veľký dôraz na vytvorenie detailnej dokumentácie klientvových požiadaviek, aby sa zistilo aké funkcie má program obsahovať. Vďaka tejto podrobnej dokumentácii bude manažment vedieť vyhradiť pre projekt potrebné financie a presne určiť čas, za ktorý by sa mal hotový produkt dostal ku klientovi. Vývojári vďaka dokumentácii presne vedia, ako má koncový produkt vyzerať a môžu začať pracovať na návrhu. V ktorom riešia aké algoritmy použiť, vedia nájsť rôzne chyby v kóde predtým, ako ho začnú implementovať, čím sa vyhnú zbytočnému prepisu kódu. Keď dokončia návrhovú fázu presúvajú sa na samotné písanie kódu a potom na testovanie, v ktorom sa preveruje či program funguje tak, ako má, aby ho mohli doručiť klientovi. Ak program funguje bez chýb dodá sa klientovi a už sa len počas používania produktu robí údržba, aby všetko fungovalo. Tu nám, ale pri tejto metóde vyskakuje problém, že ak pri testovacej fáze vybehne nejaká chyba musí sa prepísať celý program od začiatku, pretože vodopádový model sa ťažko prispôsobuje zmenám a neumožňuje sa vývojárom vracať k predošlej fáze, preto sa volá vodopádový, pretože jednotlivými fázami sa dá pohybovať len smerom z hora dole, a ak je jedna fáza ukončená nemôžeme sa k nej vrátiť. Ďalšou nevýhodou je, že po ukončení fázy, kde sa spisujú požiadavky sa s klientom už nekomunikuje, ale z môjho pohľadu je toto skôr nevýhoda pre samotného klienta nie pre vývojárov. Klient chce byť súčasťou procesu a vkladať nové funkcie do projektu, na to, ale vodopádový model nie je stavaný a tieto zmeny v požiadavkách by spôsobovali pravdepodobne veľké finančné, časové a pri najhoršom aj psychické problémy pre tím, ktorý by musel kód neustále prepisovať. Preto pri tom, aby sme vo svojom projekte mohli využiť efektívne vodopádový model musíme ho implementovať pri projektoch, ktoré majú nemeniace sa podmienky.

\section{Špirálový vývoj}\label{spiral}
Špíralový vývoj bol vytvorený Barrym Boehom v roku 1988. Je to model navrhnutý na kontrolu risku a obsahuje najlepšie praktiky z vodopádového modelu a prototypovacieho modelu. Káždá iterácia sa vola špirála ktorej výsledokm je fungujúci softvér, ktorý je dodaný klientovi na kontrolu. na začiatku každej špirály sa robí návrh pomocou postupov používaných vo vodopádovej metóde.
\section{ RAD(Rapid aplication development)}\label{RAD}
\ldots
\section{Porovnanie Jednotlivých metód medzi sebou}\label{porovnávanie}
\ldots
\section{Zhrnutie}\label{zaver}
\ldots
\bibliography{zdroje}
\bibliographystyle{plain}
\end{document}